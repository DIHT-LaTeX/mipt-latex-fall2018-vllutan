\documentclass{article}
\usepackage[utf8]{inputenc}
\usepackage[russian]{babel}

\title{Домашняя работа №1}
\author{Лутан Валерия}
\date{}

\begin{document}
\maketitle

\begin{flushright}
{\small \itshape Audi multa,

loquere pauca}
\end{flushright}
\vspace{20pt}

Это мой первый документ в системе компьютерной верстки \LaTeX . \\

\begin{center}
{\huge \sffamily <<Ура!!!>>}\\
\end{center}

А теперь формулы. {\scshape Формула}~--- краткое и точное словесное выражение, определение или же ряд математических величин, выраженный условными знаками.

\vspace{15pt}
\hspace{28pt}{\Large \bfseries Термодинамика}

Уравнение Менделеева--Клапейрона~--- уравнение состояния идеального газа, имеющее вид $pV = \nu RT$, где $p$~--- давление, $V$~--- объем, занимаемый газом, $T$~--- температура газа, $\nu$~--- количество вещества газа, а $R$~--- универсальная газовая постоянная.

\vspace{15pt}
\hspace{28pt}{\Large \bfseries Геометрия \hfill Планиметрия}

Для {\slshape плоского} треугольника со сторонами $a, b, c$ и углом $\alpha$, лежащим против стороны $a$, справедливо соотношение $$a^2 = b^2 + c^2 - 2bc\cos{\alpha},$$ из которого можно выразить косинус угла треугольника: $$\cos{\alpha} = \frac{b^2 + c^2 - a^2}{2bc}.$$ \\

Пусть $p$~--- полупериметр треугольника, тогда путем несложных преобразований можно получить, что $$ \tg{\frac{\alpha}{2}}= \sqrt{\frac{(p - b)(p - c)}{p(p - a)}},$$\\
\vspace{1cm}
На сегодня, пожалуй, хватит\dotsУдачи!

\end{document}
